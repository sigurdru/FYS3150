\documentclass[reprint, english,notitlepage,nofootinbib]{revtex4-1}  % defines the basic parameters of the document
% if you want a single-column, remove reprint

% allows special characters (including æøå)
\usepackage[utf8]{inputenc}
%\usepackage [norsk]{babel} %if you write norwegian
\usepackage[english]{babel}  %if you write english


%% note that you may need to download some of these packages manually, it depends on your setup.
%% I recommend downloading TeXMaker, because it includes a large library of the most common packages.

\usepackage{physics,amssymb}  % mathematical symbols (physics imports amsmath)
\usepackage{graphicx}         % include graphics such as plots
\usepackage{xcolor}           % set colors
\usepackage{hyperref}         % automagic cross-referencing (this is GODLIKE)
\usepackage{tikz}             % draw figures manually
\usepackage{listings}         % display code
\usepackage{subfigure}        % imports a lot of cool and useful figure commands
\usepackage{verbatim}
\usepackage{adjustbox}


% defines the color of hyperref objects
% Blending two colors:  blue!80!black  =  80% blue and 20% black
\hypersetup{ % this is just my personal choice, feel free to change things
    colorlinks,
    linkcolor={red!50!black},
    citecolor={blue!50!black},
    urlcolor={blue!80!black}}

%% Defines the style of the programming listing
%% This is actually my personal template, go ahead and change stuff if you want
\lstset{ %
	inputpath=,
	backgroundcolor=\color{white!88!black},
	basicstyle={\ttfamily\scriptsize},
	commentstyle=\color{magenta},
	language=Python,
	morekeywords={True,False},
	tabsize=4,
	stringstyle=\color{green!55!black},
	frame=single,
	keywordstyle=\color{blue},
	showstringspaces=false,
	columns=fullflexible,
	keepspaces=true}

\newcommand\numberthis{\addtocounter{equation}{1}\tag{\theequation}}
\newcommand{\ihat}{\boldsymbol{\hat{\textbf{\i}}}}
\newcommand{\jhat}{\boldsymbol{\hat{\textbf{\j}}}}
\newcommand{\khat}{\boldsymbol{\hat{\textbf{k}}}}
\newcommand{\del}[1]{\textbf{#1)}}
\newcommand{\svar}[1]{\underline{\underline{{#1}}}}
\newcommand{\vc}[1]{\mathbf{#1}}

\graphicspath{{../output/}} % search for figures in this dir



\begin{document}


\begin{titlepage}
	\begin{center}
	\textbf{Studies of phase transitions in magnetic systems}

	\vspace{0.2cm}
	Vegard Falmår and Sigurd Sørlie Rustad

	\vspace{0.5cm}
	\includegraphics[scale=0.5]{../../pictures/UIO}
	\vspace{0.8cm}

	University of Oslo\\
	Norway\\
	\today	\\
	\end{center}
	\tableofcontents
	\clearpage
\end{titlepage}

\begin{abstract}

\end{abstract}
\maketitle                              % creates the title


\section{Introduction}



\section{Theory}

\subsection*{Analytical expressions for 2x2 lattice} \label{sect:2by2Lattice}

Table \ref{tab:E_and_M_2D_lattice} shows the energy and magnetization of the 2D lattice for different spin configurations, as well as the multiplicity of each configuration. From this table we see that there are only five possible values for the energy differences $\Delta E$:
\begin{itemize}
  \item $\Delta E = \pm 16$ J for the difference between 8 J and -8 J (both ways)
  \item $\Delta E = \pm 8$ J for the difference between $\pm 8$ J and 0 J (both ways)
  \item $\Delta E = 0$ J
\end{itemize}

\begin{table}
  \input{tables/tab_E_and_M_2D_tex.txt}
  \caption{Table showing the energy, multiplicity and magnetization of different configurations of spins in a $2 \times 2$ 2D-lattice with periodic boundary conditions.}
  \label{tab:E_and_M_2D_lattice}
\end{table}



\section{Methods}

As presented in the Theory section (section \ref{sect:2by2Lattice}), we already know the energy differences in the lattice before we start the simulation. We can thus compute and store the different values of $e^{- \beta \Delta E}$ beforehand to avoid making these computations every time we update the energy.


\subsection{Boundary conditions}

We are going to simulate a 2D lattice with periodic boundary conditions. This means that the neighbour to the right of $s_N$ takes the value of $s_0$ and the neighbour to the left of $s_0$ takes the value of $s_N$.



\section{Results}

\section{Discussion}

\onecolumngrid
\vspace{1cm} % some extra space

\begin{thebibliography}{}
\bibitem[]{oppgavetekst} Department of Physics, Univeristy of Oslo, Fall semester 2020, Computational Physics I FYS3150/FYS4150, Project 3.
\bibitem[]{NASA} Ryan S. Park, Alan B. Chamberlin, NASA, 27. October 2020, https://ssd.jpl.nasa.gov/horizons.cgi\#top.

\end{thebibliography}


\end{document}
